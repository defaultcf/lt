\documentclass{beamer}
\usepackage{url}
\usepackage{animate}

\usetheme{metropolis}

\usepackage[dvipdfmx]{}

\title{パスワード管理方法論}
\author{佐藤礼於}
\date{July 14, 2018}

\begin{document}
\frame{\maketitle}

\begin{frame}{自己紹介}
  \centering \Large 佐藤 礼於

  \begin{itemize}
    \item 学部4年(実質3年)
    \item 最近はブラウザ上でサイリウムを振ってます
  \end{itemize}
\end{frame}

\begin{frame}{こんなん}
  \animategraphics[height=2.8in,autoplay,controls,loop]{12}{img/animation_}{000}{099}
\end{frame}

\frame{\centering \Large パスワード,どう管理してる?}

\begin{frame}{至るところで問われるパスワード}
  \begin{itemize}
    \item Webサービスのパスワード
    \item サーバのログインパスワード
    \item DBのログインパスワード
    \item etc...
  \end{itemize}

  どんどん増える...
\end{frame}

\begin{frame}{よくない管理方法}
  \begin{itemize}
    \item 付箋にパスワード書いてPCにペタリ
    \item 同じ短い脆弱なパスワードを使いまわし
  \end{itemize}
\end{frame}

\begin{frame}{今日に於ける一般的なパスワード管理方法}
  全てのパスワードは,英数字+記号でランダム生成16文字くらい.
  \begin{itemize}
    \item ブラウザのパスワード保存機能を使う
      \begin{itemize}
        \item Webサイトの保存ならこれで十分
        \item サーバのログインパスワードはできない
      \end{itemize}
    \item パスワード管理サービスを使う
      \begin{itemize}
        \item 1Password
        \item LastPass
        \item etc...
      \end{itemize}
  \end{itemize}
\end{frame}

\begin{frame}{ただ理想は...}
  シェルで完結させたい
\end{frame}

\begin{frame}{そこで}
  \begin{center}
    Password Store

    \url{www.passwordstore.org}
  \end{center}
\end{frame}

\begin{frame}{Password Storeの特徴}
  \begin{itemize}
    \item GPGを用いた暗号化
    \item 手元で完結する信頼感
    \item 他端末との同期はGit
  \end{itemize}
\end{frame}

\end{document}
